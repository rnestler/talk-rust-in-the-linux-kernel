\section{My Journey}

\subsection{The naïve start}

\begin{frame}[c,fragile]{Install and try it}
  It should be very straightforward
  \begin{enumerate}
    \item Verify that the ArchLinux kernel has \texttt{CONFIG\_HAVE\_RUST} set\footnote{\url{https://gitlab.archlinux.org/archlinux/packaging/packages/linux/-/blob/6.1.arch1-1/config\#L786}}
    \item Install Linux 6.1 from the testing repository
    \item Clone \url{https://github.com/Rust-for-Linux/rust-out-of-tree-module}
    \item Compile and Run
    \pause \item Profit?
  \end{enumerate}
\end{frame}

\begin{frame}[c,fragile]{Well...}
\begin{minted}[fontsize=\small]{text}
rust-out-of-tree-module (git)-[main] % make
make -C /lib/modules/`uname -r`/build M=$PWD
make[1]: Entering directory '/usr/lib/modules/6.1.0-arch1-1/build'
  RUSTC [M] ~/projects/github/Rust-for-Linux/rust-out-of-tree-module/rust_out_of_tree.o
error: target file "./rust/target.json" does not exist

make[2]: *** [scripts/Makefile.build:307: ~/projects/github/Rust-for-Linux/rust-out-of-tree-module/rust_out_of_tree.o] Error 1
make[1]: *** [Makefile:1992: ~/projects/github/Rust-for-Linux/rust-out-of-tree-module] Error 2
make[1]: Leaving directory '/usr/lib/modules/6.1.0-arch1-1/build'
make: *** [Makefile:6: default] Error 2
\end{minted}
\end{frame}

\begin{frame}[c,fragile]{RTF README}
\begin{quote}
The kernel tree (KDIR) requires the Rust metadata to be available. These are
generated during the kernel build, but may not be available for
installed/distributed kernels (the scripts that install/distribute kernel
headers etc. for the different package systems and Linux distributions are not
updated to take into account Rust support yet).
\end{quote}
README from \url{https://github.com/Rust-for-Linux/rust-out-of-tree-module}
\end{frame}

\begin{frame}[c,fragile]{Trying it in Kernel}
\begin{enumerate}
\item Clone https://github.com/Rust-for-Linux/linux
\item RTFM\footnote{\url{https://www.kernel.org/doc/html/latest/rust/quick-start.html}}
\item Check
\end{enumerate}
\end{frame}

\begin{frame}[c,fragile]{Well...}
\begin{minted}[fontsize=\footnotesize]{text}
Rust-for-Linux/linux (git)-[rust] % make LLVM=1 rustavailable
***
*** Rust compiler 'rustc' is too new. This may or may not work.
***   Your version:     1.66.0
***   Expected version: 1.62.0
***
***
*** Rust bindings generator 'bindgen' is too new. This may or may not work.
***   Your version:     0.63.0
***   Expected version: 0.56.0
***
Rust is available!
\end{minted}
\end{frame}

\begin{frame}[c,fragile]{We can fix that!}
\begin{minted}[fontsize=\footnotesize]{text}
rustup override set $(scripts/min-tool-version.sh rustc)
rustup component add rust-src
cargo install --locked --version $(scripts/min-tool-version.sh bindgen) bindgen
\end{minted}
\end{frame}

\begin{frame}[c,fragile]{Trying it in Kernel}
\begin{enumerate}
\item Switch some stuff in \texttt{make menuconfig}\footnote{In hindsight this was an important hint}
\item Compile the kernel \texttt{make LLVM=1} \rightarrow{} works
\item Switch to the kernel sources matching my running kernel and compile again \rightarrow{} works
\end{enumerate}
\end{frame}

\begin{frame}[c,fragile]{Now it just \emph{has} to work, right?}
\pause
\begin{minted}[fontsize=\footnotesize]{text}
Rust-for-Linux/rust-out-of-tree-module (git)-[main] % make KDIR=../linux LLVM=1
...
error: proc macro panicked
  --> ~/github/Rust-for-Linux/rust-out-of-tree-module/rust_out_of_tree.rs:7:1
   |
7  | / module! {
8  | |     type: RustOutOfTree,
9  | |     name: "rust_out_of_tree",
10 | |     author: "Rust for Linux Contributors",
11 | |     description: "Rust out-of-tree sample",
12 | |     license: "GPL",
13 | | }
   | |_^
   |
   = help: message: Expected byte string
\end{minted}
\end{frame}

\begin{frame}[c,fragile]{We can fix that!}
Just change every string to \texttt{b"string"} and fix the other compile errors
\footnote{\url{https://github.com/Rust-for-Linux/rust-out-of-tree-module/pull/3}}
\pause
\begin{minted}[fontsize=\footnotesize]{text}
Rust-for-Linux/rust-out-of-tree-module (git)-[main] % make KDIR=../linux LLVM=1
make -C ../linux M=$PWD
make[1]: Entering directory '~/projects/github/Rust-for-Linux/linux'
  RUSTC [M] ~/projects/github/Rust-for-Linux/rust-out-of-tree-module/rust_out_of_tree.o
  MODPOST ~/projects/github/Rust-for-Linux/rust-out-of-tree-module/Module.symvers
  LD [M]  ~/projects/github/Rust-for-Linux/rust-out-of-tree-module/rust_out_of_tree.ko
make[1]: Leaving directory '~/projects/github/Rust-for-Linux/linux'
\end{minted}
\note{
This actually downgrades the git module to the old interface. For Linux 6.2 we
reverted this fix again.
}
\end{frame}

\begin{frame}[c]{It compiles!}
  \centering
  \includegraphics[height=0.8\textheight]{img/compiling-linux-kernel-module-in-rust.jpg}
\end{frame}

\begin{frame}[c,fragile]{Now load it!}
\begin{minted}[fontsize=\footnotesize,breaklines]{text}
$ sudo insmod rust_out_of_tree.ko
insmod: ERROR: could not insert module rust_out_of_tree.ko: Invalid module format
$ dmesg | grep rust_out_of_tree
[24593.691184] rust_out_of_tree: version magic '6.1.0-arch1 SMP preempt mod_unload ' should be '6.1.0-arch1-1 SMP preempt mod_unload '
\end{minted}
\pause Alright I made a tiny mistake in defining the kernel version...
\end{frame}

\begin{frame}[c,fragile]{Now load it!}
\begin{minted}[fontsize=\footnotesize,breaklines]{text}
$ sudo insmod rust_out_of_tree.ko
insmod: ERROR: could not insert module rust_out_of_tree.ko: Unknown symbol in module
\end{minted}
\begin{minted}[fontsize=\footnotesize]{text}
$ dmesg | grep rust_out_of_treej
[25300.463334] rust_out_of_tree: Unknown symbol _RNvNtNtCsfATHBUcknU9_6kernel5print14format_strings4INFO (err -2)
[25300.463353] rust_out_of_tree: Unknown symbol _RNvNtCsfATHBUcknU9_6kernel5print11call_printk (err -2)
[25300.463371] rust_out_of_tree: Unknown symbol __rust_dealloc (err -2)
[25300.463386] rust_out_of_tree: Unknown symbol __rust_realloc (err -2)
[25300.463400] rust_out_of_tree: Unknown symbol __rust_alloc (err -2)
...
\end{minted}
\pause Hmm...
\end{frame}

\begin{frame}[c,fragile]{Investigating}
Try to use the ArchLinux config file directly and recompile
\begin{minted}[fontsize=\footnotesize]{text}
% make KDIR=../linux LLVM=1
make -C ../linux M=$PWD
make[1]: Entering directory '~/projects/github/Rust-for-Linux/linux'
  MODPOST .../Module.symvers
ERROR: modpost: "_RNvNtNtCsfATHBUcknU9_6kernel5print14format_strings4INFO" [.../rust_out_of_tree.ko] undefined!
ERROR: modpost: "_RNvNtCsfATHBUcknU9_6kernel5print11call_printk" [.../rust_out_of_tree.ko] undefined!
ERROR: modpost: "__rust_dealloc" [.../rust_out_of_tree.ko] undefined!
ERROR: modpost: "__rust_realloc" [.../rust_out_of_tree.ko] undefined!
ERROR: modpost: "__rust_alloc" [.../rust_out_of_tree.ko] undefined!
\end{minted}
\pause This looks oddly familiar!
\end{frame}

\begin{frame}[c,fragile]{The end}
\small The ArchLinux kernel doesn't have Rust support. make menuconfig
reveals, that options which conflict with CONFIG\_HAVE\_RUST=y are enabled:
\includegraphics[height=0.75\textheight]{img/make-menuconfig.png}
\end{frame}


\begin{frame}[c,fragile]{What did we learn?}
\begin{itemize}
\item Rust support needs to compile the kernel with clang/LLVM
\item The kernel only attempts to load modules which have the same version magic
\item It needs a specific \emph{stable} Rust version {\footnotesize(since it uses the
\texttt{RUSTC\_BOOTSTRAP=1} trick to enable unstable features on stable compilers)}
\item The module interface is (of course) still in flux
\item The kernel has automatic configuration variables which detect if stuff can be enabled \texttt{CONFIG\_RUST\_IS\_AVAILABLE}
\item Actual Rust support depends on a combination of multiple variables
\end{itemize}
\end{frame}

\subsection{Compiling our own kernel}

\subsection{Packing the correct metadata}

\subsection{There and Back Again}
